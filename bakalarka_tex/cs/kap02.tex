\cahpter{Analýza zdroje dat}

V této kapitole je popsán zdroj real-timových dat o polohách vozidel využívané v této práci.

\section{Přístup k datům}

Na mnohých jednáníh s kolegy ze společnosti Operator ICT bylo řečeno, že využívané vozidla vysílají data o své poleze při různých událostech. Zejména pak při brždění, rozjezdu, ale také, pro účely této práce nejdůležitější, při vyhlášení zastávky, nebo jinak každých 20 sekund.

\bigbreak

Taková data pak přímo putují k provozavoateli systému na monitorování vozidel. Ten však tato data zpracovává a posílá ke zveřejnění na platformě Golemio. Bohužel při tomto procesu zpracování se vytratí informace o události v jáké byly data pořízeny. Tedy informace o příjezdu nebo odjezdu ze zastávky je zjistitelné pouze z \gls{gps} souřadnic.

\bigbreak

Po té co jsou tyto data přeneseny do společnosti Operátor ICT by měla být zveřejněna, nicméně data ve výše popsané podobě jsou poměrně chudá, proto je k nim přidáno více atributů. Z pohledu této práce je nejzajímavější informace o vzdálenosti, kterou vozidlo urazdilo od jeho výchozí zastávky. Dále jsou přidána data o jízdních řádech a zastávkách jejichž púvodcem je \gls{ropid}.

\subsection{Dokumentace}

Na úvod je nutné poznamenat, že datová platforma je stále ve vývoji a formát dat se může měnit. S tím mohou přicháy zet určité výpadky a problémy. K jednomu takovému výpadku dosšlu při vývoji této práce, kdy po dobu 14 dnů platfomarma vůbec neodpovídala na dotazy nebo vracela prázdné datasety.

\bigbreak

Současně s využívaným datovým formátem, je nasazený pokročilý formát který obsahuje více informací a je přehledněji opraven. Nicméně při zahájení vývoje této práce nebyl k dispozici, proto jsou využívána data pouze ze starší verze.

\bigbreak

Oficiální dokumnetace datové platformy je poměrně zastaralá sama o sobě, tak že aktuální sada parametrů jí neodpovídá a neobsahuje žádné popisy dat. Proto vysvětlení jednotlivých atributů se zakládá na intuitivním pochopení nebo vyplynulo z jednání se správci platformy. V následujících kapitolách bude popsán formát dat, tak jak přichází od zdroje, a proto se může od oficiálně vystaené dokumentace lišit. A také budou popsány pouze atributy využívané v této práci nebo zajímavé pro její budoucí rozvoj.

TODO reference na dokumentaci

\bigbreak

Každá datová sada je exportována ve formátu \gls{json}. A přistupuje se k nim přes jednotné \gls{api} pomocí \gls{http} požadavku daného \gls{url} adresou a jeho hlavičkou.

\subsubsection{Pozice vozidel}

Jsou nejdůležijtější datovou sadou pro tuto práci. Jelikož se jedná o real-time data data rychle zastarávají a je nutné je velmi často aktualizaovat.

\subsubsection{Jednotlivé tripy} TODO jak se rekne trip cesky

Dále jsou k dispozici data o každém tripu. To je popis trasy vozidla, včetně zastávek a časů příjezdů a odjezdů do/z nich. Míra unikátnosti těchto tripů je předmětem dohadů a zřejmě jsou pod správou plánovačů \gsl{mhd}, nicméně můžeme s určitou mírou spolehlivosti tvrdit, že každý trip se jede nejvýše jednou za den.

\subsubsection{Zastávky}

Informace o zastávkách jsou v této práci využívány pouze k zobrazení uživateli aplikace.




















Vypozorováním zjištěno, že shape traveled je po celých 100 metrech.
