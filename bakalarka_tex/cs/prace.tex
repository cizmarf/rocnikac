%%% Hlavní soubor. Zde se definují základní parametry a odkazuje se na ostatní části. %%%

%% Verze pro jednostranný tisk:
% Okraje: levý 40mm, pravý 25mm, horní a dolní 25mm
% (ale pozor, LaTeX si sám přidává 1in)
\PassOptionsToPackage{hyphens}{url}
\documentclass[12pt,a4paper]{report}
\setlength\textwidth{145mm}
\setlength\textheight{247mm}
\setlength\oddsidemargin{15mm}
\setlength\evensidemargin{15mm}
\setlength\topmargin{0mm}
\setlength\headsep{0mm}
\setlength\headheight{0mm}
% \openright zařídí, aby následující text začínal na pravé straně knihy
\let\openright=\clearpage

%% Pokud tiskneme oboustranně:
% \documentclass[12pt,a4paper,twoside,openright]{report}
% \setlength\textwidth{145mm}
% \setlength\textheight{247mm}
% \setlength\oddsidemargin{14.2mm}
% \setlength\evensidemargin{0mm}
% \setlength\topmargin{0mm}
% \setlength\headsep{0mm}
% \setlength\headheight{0mm}
% \let\openright=\cleardoublepage

%% Vytváříme PDF/A-2u
\usepackage[a-2u]{pdfx}

%% Přepneme na českou sazbu a fonty Latin Modern
\usepackage[czech]{babel}
\usepackage{lmodern}
\usepackage[T1]{fontenc}
\usepackage{textcomp}

%% Použité kódování znaků: obvykle latin2, cp1250 nebo utf8:
\usepackage[utf8]{inputenc}

%Abbreviations
\usepackage{glossaries}

\newglossaryentry{mhd}{%
name={MHD},%
description={městská hromadná doprava}%
}

\newglossaryentry{ropid}{%
name={ROPID},%
description={Regionální organizátor pražské integrované dopravy, p. o.}%
}

\newglossaryentry{api}{%
name={API},%
description={rozhraní pro programování aplikací}%
}

\newglossaryentry{rmse}{%
name={RMSE},%
description={root-mean-square error}%
}

\newglossaryentry{osm}{%
name={OSM},%
description={OpenStreetMap}%
}

\newglossaryentry{pid}{%
name={PID},%
description={Pražská integrovaná doprava}%
}

\newglossaryentry{oop}{%
name={PID},%
description={Object Oriented Programming}%
}

\newglossaryentry{gps}{%
name={GPS},%
description={Global Position System}%
}

\newglossaryentry{geojson}{%
name={GEOJSON},%
description={standardní formát navržený pro reprezentaci jednoduchých prostorových geografických dat, specifikace: https://tools.ietf.org/html/rfc7946}%
}

\newglossaryentry{json}{%
name={JSON},%
description={JavaScript Object Notation, specifakce: https://tools.ietf.org/html/rfc8259}%
}

\newglossaryentry{utc}{%
name={UTC},%
description={Koordinovaný světový čas}%
}

\newglossaryentry{http}{%
name={HTTP},%
description={HyperText Transfer Protocol}%
}

\newglossaryentry{url}{%
name={URL},%
description={Unique Resource Link}%
}

\newglossaryentry{kapsch}{%
name={Kapsch},%
description={Kapsch, rodinná firma}%
}

\newglossaryentry{dpp}{%
name={DPP},%
description={Dopravní podnik hlavního města Prahy, a.s}%
}

\newglossaryentry{sql}{%
name={SQL},%
description={Structured Query Language}%
}

\newglossaryentry{int}{%
name={INT},%
description={Celé číslo}%
}

\newglossaryentry{dom}{%
name={DOM},%
description={Document Object Model}%
}

\newglossaryentry{html}{%
name={HTML},%
description={Hypertext Markup Language}%
}

\newglossaryentry{css}{%
name={CSS},%
description={Cascading Style Sheets}%
}

\newglossaryentry{js}{%
name={JS},%
description={JavaScript}%
}

\newglossaryentry{ajax}{%
name={AJAX},%
description={Asynchronous JavaScript and XML}%
}

\newglossaryentry{vhd}{%
name={VHD},%
description={Veřejná hromadná doprava}%
}

\newglossaryentry{idsjmk}{%
name={IDSJMK},%
description={Integrovaný dopravní systém Jihomoravského kraje}%
}

\newglossaryentry{wsgi}{%
name={WSGI},%
description={Web Server Gateway Interface}%
}

\newglossaryentry{gtfs}{%
name={GTFS},%
description={Web Server Gateway Interface}%
}

%TODO je nutne tady zkratkovat kazdy jazyk?

\makenoidxglossaries

%%% Další užitečné balíčky (jsou součástí běžných distribucí LaTeXu)
\usepackage{amsmath}        % rozšíření pro sazbu matematiky
\usepackage{amsfonts}       % matematické fonty
\usepackage{amsthm}         % sazba vět, definic apod.
\usepackage{bbding}         % balíček s nejrůznějšími symboly
			    % (čtverečky, hvězdičky, tužtičky, nůžtičky, ...)
\usepackage{bm}             % tučné symboly (příkaz \bm)
\usepackage{graphicx}       % vkládání obrázků
\usepackage{fancyvrb}       % vylepšené prostředí pro strojové písmo
\usepackage{indentfirst}    % zavede odsazení 1. odstavce kapitoly
\usepackage{natbib}         % zajištuje možnost odkazovat na literaturu
			    % stylem AUTOR (ROK), resp. AUTOR [ČÍSLO]
\usepackage[nottoc]{tocbibind} % zajistí přidání seznamu literatury,
                            % obrázků a tabulek do obsahu
\usepackage{icomma}         % inteligetní čárka v matematickém módu
\usepackage{dcolumn}        % lepší zarovnání sloupců v tabulkách
\usepackage{booktabs}       % lepší vodorovné linky v tabulkách
\usepackage{paralist}       % lepší enumerate a itemize
\usepackage{xcolor}         % barevná sazba

%%% Údaje o práci

% Název práce v jazyce práce (přesně podle zadání)
\def\NazevPrace{Analýza real-time dat vozidel městské hromadné dopravy}

% Název práce v angličtině
\def\NazevPraceEN{Analysis of real-time data of public transport vehicles}

% Jméno autora
\def\AutorPrace{Filip Čižmář}

% Rok odevzdání
\def\RokOdevzdani{2020}

% Název katedry nebo ústavu, kde byla práce oficiálně zadána
% (dle Organizační struktury MFF UK, případně plný název pracoviště mimo MFF)
\def\Katedra{Katedra softwarového inženýrství}
\def\KatedraEN{Department of Software Engineering}

% Jedná se o katedru (department) nebo o ústav (institute)?
\def\TypPracoviste{Katedra}
\def\TypPracovisteEN{Department}

% Vedoucí práce: Jméno a příjmení s~tituly
\def\Vedouci{doc. Mgr. Martin Nečaský, Ph.D.}

% Pracoviště vedoucího (opět dle Organizační struktury MFF)
\def\KatedraVedouciho{Department of Software Engineering}
\def\KatedraVedoucihoEN{Katedra softwarového inženýrství}

% Studijní program a obor
\def\StudijniProgram{Informatika}
\def\StudijniObor{SW a datové inženýrství}

% Nepovinné poděkování (vedoucímu práce, konzultantovi, tomu, kdo
% zapůjčil software, literaturu apod.)
\def\Podekovani{%
Především děkuji svému vedoucímu práce Martinu Nečaskému, který mi pomohl najít zajímavé zaměření mé práce, umožnil přístup k otevřeným datům a pomohl při vypracování.

\bigbreak

Stejně tak děkuji i pánům Janu Vlasatému a Benediku Kotmelovi, kteří mi poskytli odbornou pomoc při získávání dat z datové platformy Golemio a inspiraci pro obsah mé práce.

\bigbreak

Dále děkuji panu profesoru Jakubu Klímkovi za zapůjčení počítače za účelem získání testovacích dat.

}

% Abstrakt (doporučený rozsah cca 80-200 slov; nejedná se o zadání práce)
\def\Abstrakt{%
Tato práce se zaměřuje na analýzu dostupných otevřených real-time dat z vozidel hromadné dopravy v Praze a okolí. Jejím cílem je poskytnout základní statistické informace a na základě historických dat zlepšit odhad zpoždění spoje na trase mezi dvěma referenčními body. Jako vedlejší produkt vytvoří aplikaci pro webové rozhraní, kde zobrazí aktuální polohy spojů do mapového podkladu a rozšiřující infmace o nich. Aplikace bude aktivně interagovat s uživatelem.
}
\def\AbstraktEN{%
Abstract.
Tato práce se zaměřuje na analýzu dostupných otevřených real-time dat z vozidel hromadné dopravy v Praze a okolí. Jejím cílem je poskytnout základní statistické informace a na základě historických dat zlepšit odhad zpoždění spoje na trase mezi dvěma referenčními body. Jako vedlejší produkt vytvoří aplikaci pro webové rozhraní, kde zobrazí aktuální polohy spojů do mapového podkladu a rozšiřující infmace o nich. Aplikace bude aktivně interagovat s uživatelem.
}

% 3 až 5 klíčových slov (doporučeno), každé uzavřeno ve složených závorkách
\def\KlicovaSlova{%
{zpoždění MHD} {otevřená data} {veřejná doprava}
}
\def\KlicovaSlovaEN{%
{delay} {open data} {public transport}
}

%% Balíček hyperref, kterým jdou vyrábět klikací odkazy v PDF,
%% ale hlavně ho používáme k uložení metadat do PDF (včetně obsahu).
%% Většinu nastavítek přednastaví balíček pdfx.
\hypersetup{unicode}
\hypersetup{breaklinks=true}

%% Definice různých užitečných maker (viz popis uvnitř souboru)
\include{makra}

%% Titulní strana a různé povinné informační strany
\begin{document}
\include{titulka}

%%% Strana s automaticky generovaným obsahem bakalářské práce

\tableofcontents

%%% Jednotlivé kapitoly práce jsou pro přehlednost uloženy v samostatných souborech
\chapter*{Úvod}
\addcontentsline{toc}{chapter}{Úvod}

Městská hromadná doprava v Praze a Středočeském kraji je jeden z hlavním pilířů přepravy osob v této oblasti. Svým rozsahem a důležitostí se přímo dotýká každého z nás a její fungování do značné míry ovlivnˇuje naše konání v krátkém i dlouhém časovém horizontu.

\bigbreak

Každého cestujícího v přepravě jistě někdy trápilo zpoždění svého spoje. To člověka přivádí k myšlenkám jestli by nebylo možné určit s jakou pravidelností, pokud s nějakou, takové zpoždění vznikají. A jestli by nemohl být informován za včasu o vzniklé anomálii.

\bigbreak

Ve vymezené oblasti operuje spousta soukromých i městských přepravců. Ti kteří spadají do naší zájmové oblasti zastřešuje organizace \gls{ropid}, která objednává jednotlivé spoje. Pro tuto práci je však důležité, že tato organizace zadala jednotlivým dopravcům vysílat aktuálním polohy jejich vozů. Tato data jsou přes zprostředkovatele zveřejnˇována na pražské datové platformě zvané Golemio, jež je ve správě společností Operátor ICT, a. s., která je vlastněná hlavním městem Praha.

\bigbreak

Nicméně v době návrhu práce, kvůli právním komplikacím, nebyly k dispozici real-time data z majoritního přepravce na území Prahy Dopravní podnik hl. m. Prahy. Vzhledem k povaze této práce avšak tyto data nenabývají takové důležitosti jako data od přepravců operujících mimo Prahu. Vzhledem k tomu, že zbylí dopravci využívají převážně autobusy k přepravě cestujících, jinými způsoby dopravy se tedy zabívat nebudeme.

\bigbreak

Práce se tedy pokusí využít dostupná otevřená data k získání infomarcí o zpoždění spojů na trase. Řešení ovšem není pouze založeno na real-tim datach, ale využívá také statická data o jízdních řádech nebo zastávkách hromadné dopravy, ale také mapové podklady.

\section*{Uživatelská aplikace}

Při dispozici dat o aktuálních polohách vozidel \gls{mhd} se nabízí jejich využití tak, že budou vynesena do mapy a tím vznikne vizuálně přívětivé uživatelské prostředí pro prohlížení aktuálního stavu sítě vozidel. Proto práce navrhuje a implementuje uživatelskou aplikaci, která tyto vozidla zobrazí a bude itergagovat s uživatelem tak, že po na uživatelskou žádost zobrazí additivní infomace o daném spoji nebo vybvrané zastávce.

\bigbreak

TODO popis aplikace

TODO testovani

TODO design




Podle získaných dat se v pracovní den vypravý přibližně (TODO spocitat pocet autobusu denne) autobusových spojů.

%%% Fiktivní kapitola s ukázkami sazby

\chapter{Analýza problému a jeho řešení}

V této kapitole je detailně popsán problém a způsoby jeho navrženého a současného řešení.

\section{Úvod}

Spoje které zajišťují hromadnou dopravu jezdí podle jízdních řádů, které definují jejich trasu. Trasa se udává sekvencí projíždících zastávek, časy příjezdu a odjezdu do, resp. z těchto zastávek a vzdáleností zastávek od výchozího bodu spoje. Tyto zastávky jsou zpravidla jediné refenční body u kterých je možno zjistit skutečné zpoždění, nebo předjetí (dále uvažováno jako zpoždění se zápornou hodnotou). Dále jsou součástí jízdních řádů také velice detailní nákresy tras každého spoje, formou lomené čáry definovanou posloupností souřadnic, kde každý bod je doplněn o jeho vzdálenost od výchozího bodu spoje.

\bigbreak

Délka trasy mezi dvěma refernčními body nezříka dosahuje i několika desítek kilometrů\footnote{Podle dat pro spoje jedoucí v 20. 2. 2020 je medián vzdálesnotí mezi zastávkama, mezi kterýma projede alesponˇ jeden spoj denně 943 m. Průjezdů mezi zatávkami ve vzdálenosti více než 10 kilometrů je 784, přiřičemž průjezdů mezi zastávkami ve vzdálenosti alesponˇ 2 km je přibližně 15000}. Na těchto úsecích mohou vznikat mimořáné události, které se dají predikovat jen s těží. Nicméně ve většině případů je průběh jízdy ovlivněn pouze obvyklým provozem v dané denní době.

Detailní rozbor počtu průjezdů mezi zastávkami v daných vzdálenostech je vidět na grafu \ref{fig:stop_distances_result}. Kde průjezdem se myslí každý jednotlivý průjezd vozidla mezi danou dvojcí zastávek v daný den. Data jsou platná pro spoje jedoucí v 20. 2. 2020.

\begin{figure}
  \includegraphics[width=\linewidth]{../img/stop_distances_plot_2020-02-20.png}
  \caption{Graf počtu úseků mezi následujícími zastávkami a vzdálenotí mezi nimi.}
  \label{fig:stop_distances_result}
\end{figure}

\section{Popis problému odhadu zpoždění}

Řešený problém se týká případu, kdy vozidlo projíždí mezi dvěma referenčními body a tato trasa má části, ve kterých vozidlo jede různou rychlostí. Např. vozidlo při vyjíždění z města jede mnohem pomaleji než při jízdě mezi městy. Takových úseků, na kterých se rychlost jízdy liší může být na trase více a nedají se všechny jednoduše detekovat.

\bigbreak

Tato Práce tedy modeluje profily jízd mezi referenčními body. A na základě toho zpřesnit odhad zpoždění. Tento odhad by měl být mnohem přesnější než současné odhady, které odpovídají tomu, že vozidlo jede konstantní rychlostí po celou dobu jízdy. Nebo je takové možné brát jako aktuální zpoždění spoje poslední změřené zpoždění při průjezdu nějak7ch referenčním bodem (zastávkou, nebo např. pro tramvaje se používají návěstidla).

\bigbreak

Přidaná hodnota je tedy v tom, že Práce navrhne takové modely, které nebudou penezalizovat zvyšováním zpožděním za pomalou jízdu v úsecích, které se pomaleji projždějí vždy. A také naopak zvýhodnˇovat snížením zpožděním za rychlou jízdu v úsecích, které se projíždějí rychle. Pokud bychom se tedy podívali na změny zpoždění na trase mezi dvěma referečními body, v ideálním případně by měli být nulové.

\bigbreak

Celé ilustrováno na příkladě.

\bigbreak

Pro řešení toho typu spoždění stačí navrhnout systém na odhat zpoždění v půběhu jízdy mezi referenčními body z historikých dat jízd.

\bigbreak

Tento odhad změny zpoždění na trase mezi dvěma referenčními body je nutné počítat v co nejkratším čase tak, aby cestující byli dobře informování o stavu jejich spoje a mohli tyto informace využít např. při dobíhání spoje. A proto je potřeba zpracovávat data okamžitě po jejich vydání, spočítat odhad zpoždění a vystavit tato data veřejně. Vzhledem k tomu, že tato data velmi rychle zastarávají je nutné provádět tento proces co možná nejrychleji\footnote{Průměrná doba jízdy spoje mezi zastávkami je cca 5 min. Rozložení počtu úseků mezi zastávekami k délce jízdy mezi nimi je závislé a podobné rozložení vůči vzdálenosti ilustrované na grafu\ref{fig:stop_distances_result}.}.

\bigbreak

Pro vyloučení všech pochybností je hodno uvést, že se naše Práce nesnaží předpovědět zpoždění, které spoj může nabrat vzhledem k dosavadnímu průběhu trasy. Tedy např. nijak nezohlednˇuje to, že spoj právě stojí v mimořádné koloně a dalo by se tedy předpokládat, že zpoždění bude rychle růst i v následujících minutách. Ale naopak Práce se snaží odhadnou zpoždění v danném bodě na trase vzhledem k obvyklému profilu jízdy. Tedy např. pokud by výše uvažavaná kolona byla pravidelná Práce ji zohlední ve statistických modelech.

\subsection{Současná řešení}

Takový algoritmus na odhat aktuálního zpoždění mezi dvěma referenčními body již exituje a je zakomponován v systému, ze kterého se čerpají data pro tuto práci. (Detailní popis dat uveden v kapitole ~\ref{chapter:TODO later}.) Nicméně nezohledňuje základní parametry průběhu trasy. Tento algoritmus nahlíží na postup vozidla na trase jako na lineární funkci vůči času. Je ovšem zřejmé, že rychlost vozidel není konstantní, neboli doba jízdy není linárně závislá na ujeté vzdálenosti.

\section{Analýza požadavků na uživatelskou aplikaci}

Součástí práce je i vizualizace spočítaných dat. Jinými slovy nástroj umožňující přístup uživatelů ke spočítaným datům.

\subsubsection{Funkční požadavky}

\begin{itemize}
	\item Aplikace vykreslí interaktivní mapu Prahy a širšího okolí, kterou bude možné posouvat či zoomovat. V této mapě budou zobrazeny jednotlivé vozidla na aktuálních pozicích a budou se automaticky posouvat po mapě, tak jak se pohybují ve skutečnosti.

	\item Po kliknutí na vozidlo se zobrazí jeho celá trasa včetně zastávek a jeho dopočítaného zpoždění.

	\item Po kliknutí na zastávku se zobrazí seznam spojů, které budou projíždět vybranou zastávkou a jejich trasy se vykreslí do mapy.

	\item Celá aplikace bude postavena na principu server -- client. Tedy serverová strana se postará o přístup k otevřeným datům o vozidlech a jejich uložení a také obsluhu požadavků klienta. Klientská část bude webová stránka poskytující služby popsané výše. Měla by být schopná zobrazit řádově tisíce vozidel.
\end{itemize}

\subsubsection{Nefunkční požadavky}

\begin{itemize}
	\item Serverová část bude napsaná v jazyce Python 3.

	\item Webová část bude napsaná pomocí jazyků pro webové technologie, převážně v JavaScriptu.

	\item Pro vykleslení mapy bude využita služba Mapbox.

	\item Ukládání dat na serverové straně bude řešeno MySQL databází.

	\item Pro algoritmus odhadu zpoždění na zákldě historických dat budou využity různé knihovny pro jazyk Python 3. Zejména pak scikit-learn a alphashape.

\end{itemize}

\subsubsection{Proces běhu aplikace}

Jak je již zmíněno aplikace bude využívat historická data, tedy bude nutné nechat aplikaci tato data nějakou dobu sbírat. Pro efektivní odhady by bylo vhodné mít uložené historické polohy vozidel alespoň z uplynulých několika týdnů.

\bigbreak

Avšak již v průběhu sběru dat může aplikace poskytovat základní službu a to vizualizování vozidel v mapě.

\subsection{Poskytovatelé mapových podkladů}

K takovému účelu nejlépe poslouží vykreslení aktuálních poloh vozidel do mapy, kde se po vyžádání uživatelem tyto data zobrazí.

\bigbreak

Za účelem vytvoření dostatečně přívětivé uživatelské aplikace je nezbytné využít některého z poskytovatelů mapových podkladů a zanést do něj získané informace.

\bigbreak

Jedním z těchto poskytovatelů je společnost Google, která má propracované mapové podklady a prostřednictvím služby Google Maps poskytuje pro tuto práci požadovanou službu. Další platformou je Mapbox, který poskytuje velmi podobné služby jako Google Maps. Nicméně narozdíl od Googlu využívá jako mapový podklad \gls{osm} {otevřená geografické data}. Protože smyslem práce je v co největší míře využít otevřená data je žádoucí využít právě Mapbox.

\bigbreak

TODO dokumentace mapbox, zeptat se jestli je to vubec nutne rozebirat

\subsection{Současná řešení}

Vizualizaci vozidel \gls{vhd} do mapy již nabízí několik portálů. Všechny jsou však poměrně strohé.

\subsubsection{Golemio}

Takovou mapu zobrazuje i samotný provozavatel datové platformy. Nicméně nejsou zde vidět ani čísla linek zobrazených autobusů, natož pak nějaké další informace.

\begin{figure}
  \includegraphics[width=\linewidth]{../img/golemio_mapa.png}
  \caption{Mapa z golemio.cz.}
  \label{fig:golemio_result}
\end{figure}

\subsubsection{Tram-bus}

Dalším poskytovatelem je portál tram-bus, který si vede o něco lépe. Ukazuje směr jízdy vozidel, čísla linek a po kliknutí informace o zpoždění a nejbližší zastávky. Pozn.: na mapě již jsou vidět spoje \gls{dpp}, protože v době psaní této práce již byly data veřejné.

\begin{figure}
  \includegraphics[width=\linewidth]{../img/tram-bus_mapa.png}
  \caption{Mapa z www.tram-bus.cz.}
  \label{fig:tram-bus_result}
\end{figure}

\subsubsection{\gls{idsjmk}}

Mimo Prahu je velice pěkně udělaná aplikace pro zobrazení vozidel \gls{idsjmk} (Integrovaný dopravní systém Jihomoravského kraje). Ten ihned po načtení stránky zobrazuje všechny dobravní prostředky, tedy tramvaje, autobusy a vlaky vše s čísly linek. Dále pak umožňuje po kliknutí na vybraný spoj zobrazit více informací včetně jízdního řádu.

\bigbreak

Tato aplikace je po vizuální i funkční stránce dobrou inspirací pro tvorbu aplikace v této práci.

\begin{figure}
  \includegraphics[width=\linewidth]{../img/idsjmk_mapa.png}
  \caption{Mapa z mapa.idsjmk.cz.}
  \label{fig:idsjmk_result}
\end{figure}

%%% Fiktivní kapitola s ukázkami sazby

\chapter{Návrh řešení}

V této kapitole je popsáno technické řešní uvedených problémů.

\section{Odhad zpoždění}

\subsection{Funkční požadavky}

Popsaný odhad změny zpoždění na trase mezi dvěma referenčními body je nutné počítat v co nejkratším čase tak, aby cestující byli dobře informování o stavu jejich spoje a mohli tyto informace využít např. při dobíhání spoje. A proto je potřeba zpracovávat data okamžitě po jejich vydání, spočítat odhad zpoždění a vystavit tato data veřejně. Vzhledem k tomu, že tato data velmi rychle zastarávají je nutné provádět tento proces co možná nejrychleji\footnote{Průměrná doba jízdy spoje mezi zastávkami je cca 5 min. Rozložení počtu úseků mezi zastávekami k délce jízdy mezi nimi je závislé a podobné rozložení vůči vzdálenosti ilustrované na grafu\ref{fig:stop_distances_result}.}.

\bigbreak

Data o polohách vozidel VHD v Datové platfomě jsou aktualizována každých 20 sekund\footnote{Řečeno zaměstnancem OICT na schůzce 4. 5. 2019}. Navíc neplatné (aktualizované) data o polohách se již neposílají. Tedy pro minimalizaci rychlosti zastarávání dat a získání všech existujících vzorků dat o polohách je nutné data stahovat nejpozději každých 20 sekund.

\bigbreak



\section{Vizualizace dat}

\subsection{Funkční požadavky}

Součástí práce je i vizualizace spočítaných dat.

\ bigbreak

To bude provedeno formou front endové aplikace, která zobrazuje mapu a do ní zanáší data o vozidlech VHD. Funkční požadavky této aplikace jsou inspirované již existujícími řešeními tohoto problému, ty jsou dále rozebrány v kapitole \ref{chapter:soucasna_reseni_front_end}

\subsection{Funkční požadavky}

\begin{itemize}
	\item

	\item Aplikace vykreslí interaktivní mapu Prahy a širšího okolí, kterou bude možné posouvat či zoomovat. V této mapě budou zobrazeny vozidla na aktuálních pozicích a budou se automaticky posouvat po mapě, tak jak se pohybují ve skutečnosti.

	\item Po kliknutí na vozidlo se zobrazí jeho celá trasa včetně zastávek a jeho dopočítaného zpoždění.

	\item Po kliknutí na zastávku se zobrazí seznam spojů, které budou projíždět vybranou zastávkou a jejich trasy se vykreslí do mapy.

	\item Celá aplikace bude postavena na principu server -- client. Tedy serverová strana se postará o přístup k otevřeným datům o vozidlech a jejich uložení a také obsluhu požadavků klienta. Klientská část bude webová stránka poskytující služby popsané výše. Měla by být schopná zobrazit řádově tisíce vozidel.
\end{itemize}

\subsubsection{Nefunkční požadavky}

\begin{itemize}
	\item Serverová část bude napsaná v jazyce Python 3.

	\item Webová část bude napsaná pomocí jazyků pro webové technologie, převážně v JavaScriptu.

	\item Pro vykleslení mapy bude využita služba Mapbox.

	\item Ukládání dat na serverové straně bude řešeno MySQL databází.

	\item Pro algoritmus odhadu zpoždění na zákldě historických dat budou využity různé knihovny pro jazyk Python 3. Zejména pak scikit-learn a alphashape.

\end{itemize}

\subsubsection{Proces běhu aplikace}

Jak je již zmíněno aplikace bude využívat historická data, tedy bude nutné nechat aplikaci tato data nějakou dobu sbírat. Pro efektivní odhady by bylo vhodné mít uložené historické polohy vozidel alespoň z uplynulých několika týdnů.

\bigbreak

Avšak již v průběhu sběru dat může aplikace poskytovat základní službu a to vizualizování vozidel v mapě.

\subsection{Poskytovatelé mapových podkladů}

K takovému účelu nejlépe poslouží vykreslení aktuálních poloh vozidel do mapy, kde se po vyžádání uživatelem tyto data zobrazí.

\bigbreak

Za účelem vytvoření dostatečně přívětivé uživatelské aplikace je nezbytné využít některého z poskytovatelů mapových podkladů a zanést do něj získané informace.

\bigbreak

Jedním z těchto poskytovatelů je společnost Google, která má propracované mapové podklady a prostřednictvím služby Google Maps poskytuje pro tuto práci požadovanou službu. Další platformou je Mapbox, který poskytuje velmi podobné služby jako Google Maps. Nicméně narozdíl od Googlu využívá jako mapový podklad \gls{osm} {otevřená geografické data}. Protože smyslem práce je v co největší míře využít otevřená data je žádoucí využít právě Mapbox.

\bigbreak

TODO dokumentace mapbox, zeptat se jestli je to vubec nutne rozebirat

\subsection{Současná řešení} \label{chapter:soucasna_reseni_front_end}

Vizualizaci vozidel \gls{vhd} do mapy již nabízí několik portálů. Všechny jsou však poměrně strohé.

\subsubsection{Golemio}

Takovou mapu zobrazuje i samotný provozavatel datové platformy. Nicméně nejsou zde vidět ani čísla linek zobrazených autobusů, natož pak nějaké další informace.

\begin{figure}
  \includegraphics[width=\linewidth]{../img/golemio_mapa.png}
  \caption{Mapa z golemio.cz.}
  \label{fig:golemio_result}
\end{figure}

\subsubsection{Tram-bus}

Dalším poskytovatelem je portál tram-bus, který si vede o něco lépe. Ukazuje směr jízdy vozidel, čísla linek a po kliknutí informace o zpoždění a nejbližší zastávky. Pozn.: na mapě již jsou vidět spoje \gls{dpp}, protože v době psaní této práce již byly data veřejné.

\begin{figure}
  \includegraphics[width=\linewidth]{../img/tram-bus_mapa.png}
  \caption{Mapa z www.tram-bus.cz.}
  \label{fig:tram-bus_result}
\end{figure}

\subsubsection{\gls{idsjmk}}

Mimo Prahu je velice pěkně udělaná aplikace pro zobrazení vozidel \gls{idsjmk} (Integrovaný dopravní systém Jihomoravského kraje). Ten ihned po načtení stránky zobrazuje všechny dobravní prostředky, tedy tramvaje, autobusy a vlaky vše s čísly linek. Dále pak umožňuje po kliknutí na vybraný spoj zobrazit více informací včetně jízdního řádu.

\bigbreak

Tato aplikace je po vizuální i funkční stránce dobrou inspirací pro tvorbu aplikace v této práci.

\begin{figure}
  \includegraphics[width=\linewidth]{../img/idsjmk_mapa.png}
  \caption{Mapa z mapa.idsjmk.cz.}
  \label{fig:idsjmk_result}
\end{figure}

\newcommand{\documentationAtt}[2] {#1 #2}

\chapter{Analýza zdroje dat}

V této kapitole je popsán zdroj real-timových dat o polohách vozidel využívané v této práci.

\section{Přístup k datům}

Na mnohých jednáníh s kolegy ze společnosti Operator ICT bylo řečeno, že využívané vozidla vysílají data o své poleze při různých událostech. Zejména pak při brždění, rozjezdu, ale také, pro účely této práce nejdůležitější, při vyhlášení zastávky, nebo jinak každých 20 sekund.

\bigbreak

Taková data pak přímo putují k provozavoateli systému na monitorování vozidel, kterým je společnost \gls{chaps} jakožto partner \gls{dpp}. Ten však tato data zpracovává a posílá ke zveřejnění na platformě Golemio. Bohužel při tomto procesu zpracování se vytratí informace o události v jáké byly data pořízeny. Tedy informace o příjezdu nebo odjezdu ze zastávky jsou zjistitelné pouze z \gls{gps} souřadnic.

\bigbreak

Po té co jsou tyto data přeneseny do společnosti Operátor ICT by měla být zveřejněna, nicméně data ve výše popsané podobě jsou poměrně chudá, proto je k nim přidáno více atributů. Z pohledu této práce je nejzajímavější informace o vzdálenosti, kterou vozidlo urazdilo od jeho výchozí zastávky. Dále jsou přidána data o jízdních řádech a zastávkách jejichž púvodcem je \gls{ropid}.

\subsection{Dokumentace}

Na úvod je nutné poznamenat, že datová platforma je stále ve vývoji a formát dat se může měnit. S tím mohou přicházet určité výpadky a problémy. K jednomu takovému výpadku došlu i při vývoji této práce, kdy po dobu 14 dnů platfomarma vůbec neodpovídala na dotazy nebo vracela prázdné datasety.

\bigbreak

Současně s využívaným datovým formátem, je nasazený pokročilý formát který obsahuje více informací a je přehledněji opraven. Nicméně při zahájení vývoje této práce nebyl k dispozici, proto jsou využívána data pouze ze starší verze.

\bigbreak

Oficiální dokumnetace datové platformy je poměrně zastaralá sama o sobě, tak že aktuální sada parametrů jí neodpovídá a neobsahuje žádné popisy dat. Proto vysvětlení jednotlivých atributů se zakládá na intuitivním pochopení nebo vyplynulo z jednání se správci platformy. V následujících kapitolách bude popsán formát dat, tak jak přichází od zdroje, a proto se může od oficiálně vystaené dokumentace lišit. A také budou popsány pouze atributy využívané v této práci nebo zajímavé pro její budoucí rozvoj.

TODO reference na dokumentaci

\bigbreak

Každá datová sada je exportována ve formátu \gls{geojson} pokud se jedná o geografická data, nebo jinak ve formátu \gls{json}. A přistupuje se k nim přes jednotné \gls{api} pomocí \gls{http} požadavku daného \gls{url} adresou a jeho hlavičkou.

TODO reference na specifikace geojson https://tools.ietf.org/html/rfc7946

\bigbreak

Ačkoli se dokumentace tváří tak, že data jsou exportována ve formátech \gls{json} nebo \gls{geojson}, většinou formát dat není přesně podle specifikace těchto formátů. Například může být uveden atribut \verb"wheelchair_accessible", který je typu \verb"bool" a je nastaven na hodnotu \verb"True", nicmně podle specifikace se tyto hodnoty píší s malým písmenem. Pro tuto práci to sice nepředstavuje komplikaci, protože tento atribut není potřeba, ale mohlo by se stát, že některé parsery \gls{json}u vyhodnotí řetězec jako nevalidní a skončí chybou.

\bigbreak

TODO Celá datová platforma Golemio je pojatá jako Open Source projekt.

\subsubsection{Pozice vozidel}

Jsou nejdůležijtější datovou sadou pro tuto práci. Jelikož se jedná o real-time data, data rychle zastarávají a je nutné je velmi často aktualizaovat.

\begin{itemize}
	\item \documentationAtt{coordinates}{aktuální \gls{gps} souřadnice vozidla}

	\item \documentationAtt{origin\_timestamp}{čas zachycení pozice vozidla, v časovém pásmu \gls{utc}}

	\item \documentationAtt{gtfs\_trip\_id}{unikátní identifik8tor tripu pro spárování s jízdním řádem}

	\item \documentationAtt{gtfs\_shape\_dist\_traveled}{vzdálenost vozidla uražená od začátku tripu v metrech}

	\item \documentationAtt{delay\_stop\_departure}{zpoždění zachycené při odjezdu z poslední projeté zastávky v sekundách}
\end{itemize}

\subsubsection{Jednotlivé tripy}

TODO jak se rekne trip cesky

Dále jsou k dispozici data o každém tripu. To je popis trasy vozidla, včetně zastávek a časů příjezdů a odjezdů do/z nich. Také může být vyžádáno k informacím o tripu připojit celý shape trasy, tj. lomená čára kopírující celo trasu daného tripu po povrchu Země.

\bigbreak

 Míra unikátnosti těchto tripů je předmětem dohadů a zřejmě jsou pod správou plánovačů \gls{mhd}, nicméně můžeme s určitou mírou spolehlivosti tvrdit, že každý trip se jede nejvýše jednou za den.

\begin{itemize}
	\item \documentationAtt{trip\_headsign}{nápis na čele vozidla, typicky cílová stanice}

	\item \documentationAtt{route\_id}{číslo linky}

	\item \documentationAtt{trip\_id}{unikátní identifikátor tripu pro spárování s real-time daty, pravděpodobně odpovídá atributu \verb"gtfs\_trip\_id"}
\end{itemize}

Navíc s každým tripem může být vyžádáno zaslání seznamu zastávek, kterýma projíždí. Po té se obdrží tento seznam s kompletními informacemi o zastávkách, tedy má stejnou informační hodnotu jako samostatný dotaz na zastávky. Navíc je každá zastávka doplněna o informace vázající se k danému tripu.

\subsubsection{Zastávky}

\begin{itemize}
	\item \documentationAtt{arrival\_time}{čas příjezdu spoje do zastávky}

	\item \documentationAtt{departure\_time}{čas odjezdu spoje do zastávky}

	\item \documentationAtt{shape\_dist\_traveled}{vzdálenost zastávky na trase od výchozího bodu daného tripuv metrech}

	\item \documentationAtt{stop\_id}{unikátní indetifikátor zastávky}

	\item \documentationAtt{coordinates}{\gls{gps} souřadnice zastávky, často \verb"None", je třeba využít atributy \verb"stop\_lat" a \verb"stop\_lon"}

	\item \documentationAtt{stop\_name}{název zastávky}
\end{itemize}



















TODO Vypozorováním zjištěno, že shape traveled je po celých 100 metrech.

%%% Fiktivní kapitola s ukázkami tabulek, obrázků a kódu

\chapter{Implementace}






\section{Modely}


Pro spočítání polynomiálního modelu se využívá knihovna sklearn konkrétně algoritmus zvaný Rigde, který sám o sobě hledá linární závislosti, nicméně vstupní hodnoty jsou mezi sebou náležitě pronásobeny tak, aby simulovali polynomiální funkci. Toho se dosáhne pomocí funkce PolynomialFeatures. Optimální stupeň polynomu se zjistí spočítáním modelu pro každý stupeň v rozumných mezích a nakonec se zvolí ten s nejmenší chybou. (TODO jak moc detailně se ma toto popisovat?)






\section{Vizualizace dat}

Data budou zobrazovány pomocí webové aplikace (klientská část) a ta bude stahovat data ze serverové části. Komunikační mapa ilustrující propojení těchto částí je zobrazena na diagramu \ref{fig:design_diagram}.

\subsection{Klientská část}

Webová aplikace bude napsána pomocí jazyků a nástrojů vhodných pro vývoj webových aplikací. Používáme tedy značkovací jazyk \gls{html} pro strukturu samotné webové stránky, pro stylování objektů je použit jazyk \gls{css}. Hlavní vlastnosti stránky, jako je zobrazení entit do mapy je použitý jazyk \gls{js}, zejména pak jeho možností pro zacházení s \gls{dom} elementy. Pro připojení a načítání dat ze serveru se používá technologie \gls{ajax}ových dotazů.

\bigbreak

Koncepce klientské aplikace je taková, že žádná data nezpracovává ani nepřepočítává a zobrazuje jen data taková, která obdržela od serverové strany typicky ve formátu \gls{geojson}. Pro aktualizi dat je potřeba vyvolat nový dotaz, typycky se stejnými parametry.

\bigbreak

Webová aplikace bude v pravidelných intervalech aktualizovat obraz všech vozidel. Dále pak bude reagovat na uživatelské vstupy v podobě klikání na vybrané elementy. Ty potom vykreslí do mapy odlišeně nebo stáhne přídavná data k zobrazení.

\subsubsection{Mapbox API}

Nejprve si popišme jaké funkce budeme využívat z knihovny Mapbox.

\bigbreak

Prostředí Mapbox je široce využívaný multiplatformový nástroj pro zobrazení mapového podkladu a umožňuje do něj zanést širokou škálu různých geometrických útvarů. Tak že mapové prostředí intuitivně interaguje s uživatelem a vývojáři mohou využití jednoduchécho \gls{api} pro zobrazení žádoucích dat do mapy.

\bigbreak

Webová aplikace této práce využívá naprosto základní funkcionality, které mapbox přináší.  Popis jejich využití včetně načtení prostředí Mapboxu do webové stránky za předpokladu, že jsou splněny základní \gls{html} požadavky webové stránky je následující.

\bigbreak

Rozhraní se do webové stránky importuje pomocí:

\begin{code}[frame=none]
<script src='https://api.tiles.mapbox.com/
  mapbox-gl-js/v1.4.0/mapbox-gl.js'></script>
<link href='https://api.tiles.mapbox.com/
  mapbox-gl-js/v1.4.0/mapbox-gl.css' rel='stylesheet' />
\end{code}

\bigbreak

Dále je potřeba vytvořit element s identifikátorem webové stránky, kde bude mapa zobrazena.

\bigbreak

Po naiportovéní je v JavaScriptu k dispozici knihovna jménem \verb-mapboxgl- pomocí, které se ovládá celé mapové prostředí. Tedy je tedˇ možné vytvořit samotnou mapu.

\begin{code}[frame=none]
var map = new mapboxgl.Map({
  container: 'map', // identifikátor HTML elementu
  style: 'mapbox://styles/mapbox/streets-v11',
  center: [14.42, 50.08], // střed mapy při inicializaci [lng, lat]
    zoom: 10 // zoom při inicializaci
});
\end{code}

Nyní stačí jen vytvořit \gls{html} element za pomocí \gls{js} a po té může být přidám do mapy následující funkcí. Nyní se nám již takový element zobrazuje v mapě na zvolených souřadnicích.

\begin{code}[frame=none]
new mapboxgl.Marker(element)
  .setLngLat([Lng, Lat]) // zeměpisná výška a šířka
    umítění elementu
  .addTo(map);
\end{code}

Pro vykreslení složitějších objektů, jako je třeba lomená čára se využívá funkce \verb-addLayer-. Tato funkce přijímá data ve formátu \gls{geojson} tedy není třeba dělat žádnou trasnformaci dat.

\begin{code}[frame=none]
map.addLayer({
  "id": id, // identifikátor vrstvy
  "type": "line", // geometrický útvar k zobrazení
  "source": {
    "type": "geojson", // formát zdrojových dat
    "data": data // zdroj dat
  },
  "paint": {
    "line-color": "#BF93E4", // barva
    "line-width": 5 // šířka
  }
});
\end{code}

K manipulaci s objekty typu \verb-Layer- se používají následující funkce.

\begin{code}[frame=none]
map.getLayer(id);
map.removeLayer(id);
\end{code}

To je vše co potřebujeme k naplnění cíle vizualizace dat. Autobus na mapě budeme reprezentovat kolečkem s číslem spoje a zastávku jako špendlík, toto jsou \gls{html} elementy. Lomené čáry trasy spoje vykreslíme jako vrstvu funkcí \verb-addLayer-.

\subsubsection{Běh aplikace}

Se serverovou částí se komunikuje pomocí \verb-GET- requestů a server vrací \gls{json}onové soubory. Webová aplikace používá knihovnu na parsování tohoto formátu a tedy můžeme se k nim chovat jako k mapám.

\bigbreak

Po inicializaci prostředí Mapboxu popsanou výše následuje inicializace naší aplikace. Především se pak spustí smyčka aktualizující aktuální polohy vozidel.

\begin{code}[frame=none]
var vehicles = new Set(); // elementy vozidel v mape
var active_trips = {}; // vybraná vozidla
var vehicles_elements = {}; // html elementy vozidel
var no_stop_chosen = true; // indikátor vybrání zastávky

// inicializační stažení poloh vozidel
getFileByAJAXreq("vehicles_positions", showBusesOnMap);

// hlavní smyčka
window.setInterval(function(){
getFileByAJAXreq("vehicles_positions", showBusesOnMap);

// aktualizace ocasu všech vybraný vozidel
for (var trip in active_trips){
  active_trips[trip].update_tail();
}
}, 10000);
\end{code}

Po načtení poloh vozidel probíhá jejich vykreslování do mapy. Nejprve se odstraní z mapy všechny staré vozidla a pro každé nové vozdilo se konstruuje nový \gls{html} element. Každý tento element reprezentující vozidlo poslouchá na kliknutí.

\bigbreak

Tedy pokud vozidlo není vybráno vytvoří se nový element a následně se vykreslí do mapy a zárovenˇ se stáhnou další informace o vozidle, které se též následně zobrazují. Jsou jimi trasa vozidla, jízdní řád a zpoždění. Vybrané vozidlo se v ko'du reprezentuje vlastní třídou \verb-Active_trip-, každá instace této třídy se přidá do promněné \verb-active_trips-. Tato třída pak obsahuje metody obstarávající zobrazení dalších informací, stejně tak jejich odstranění nebo aktualizaci.

\bigbreak

Pokud vozidlo již bylo vybráno jednoduše se odstraní z množiny vybraných vozidel \verb-active_trips- a z mapy.

\bigbreak

V případě, že je nějaké vozidlo vybráno, zobrazují se i jeho zastávky. Každá zobrazená zastávka reaguje na kliknutí a na přejetí myší. Po kliknutí se vyberou všechny spoje projíždějící zastávkou a po přejetí myší se zobrazí název zastávky. Tyto funkcionality se \gls{html} elementu přiřadí následujícím ko'dem.

\begin{code}[frame=none]
// zobrazí všechny spoje projíždějící zastávkou
el_c.addEventListener('click', function() {
  no_stop_chosen = false;
  show_trips_by_stop(getFileByAJAXreqNoCallback(
    "trips_by_stop." + marker.name
  ), marker.name);
});

// přidá do mapy název zastávky po najetí myší
el_c.addEventListener("mouseover", function(){
  var el_s = document.createElement('div');
  el_s.innerText = marker.name;
  el_s.setAttribute("class", "stop_pin_sign");
  new mapboxgl.Marker(el_s)
    .setLngLat(marker.geometry.coordinates)
    .addTo(map);
});

// odebere všechny názevy zastávek z mapy po vyjetí myši
el_c.addEventListener("mouseout", function(){
  var signs = document.getElementsByClassName('stop_pin_sign');
  while(signs[0]) {
    signs[0].parentNode.removeChild(signs[0]);
  }
});
\end{code}

Vybraná vozidla podle zastávky se pak chovají stejně jako když je vybrané pouze jedno vozidlo. Tedy do promněné \verb-active_trips- se vloží více vozidel.

\subsection{Serverová část}

Tak jak je řečeno příchozí požadavky od klineta jsou odpovídáný skriptem na serverové straně. Který je napojený na databázi a z ní extrahuje potřebná data.

\bigbreak

Data jsou posílána v textové podobě ve formátu \gls{geojson}, které skrip konstruuje z dat získaných z databáze.

\bigbreak

Server reaguje na 4 typy požadavků:

\begin{itemize}
	\item \verb-get_vehicle_positions- vrátí aktuální polohy všech vozidel,

	\item \verb-get_tail.id_trip- vrátí lomenou čáru popisující pohyb vozidla v uplynulých $n$ minutách, vozidla podle identifikátoru jízdy,

	\item \verb-get_shape.id_trip- vrátí lomenou čáru popisující trasu spoje podle id spoje, vozidla podle identifikátoru jízdy,

	\item \verb-get_stops.id_trip- vrátí seznam zastávek pro spoj podle jeho id, vozidla podle identifikátoru jízdy.
\end{itemize}

Celý server je stejně jako jádro systému naprogramováno v jazyce Python3.

\subsubsection{Server knihovna}

Server je naprogramovám pomocí Pythoní knihovny \verb-simple_server-, která slouží pouze k debugování, jak se píše v její dokumentaci\footnote{$https://docs.python.org/3/library/wsgiref.html\#module-wsgiref.simple_server$}. Protože se nepočítá s reálným nasezením této aplikace, není potřeba programovat robustní server. Pro demonstrační účely je však toto řešení dostatečné.

\bigbreak

Vytvoření serveru pomocí této knihovny se v jazyce Python3 udělá následovně.

\begin{code}[frame=none]
httpd = make_server("", self.PORT, self.server)
thread = threading.Thread(target=httpd.serve_forever)
thread.start()
\end{code}

Kde \verb-server- je funkce, která je vždy volána když server obdrží dotaz. Upozorněme, že tento způsob startu serveru je odlišný od popisu v dokumentaci.

\bigbreak

Volaná funkce \verb-server- je kompletní gls{WSGI} aplikace, jež příjmá argumenty \verb-environ-, což je dotaz a atribut \verb-start-response-, který reprezentuje hlavičku odpovědi. Dále se v této funkci nachází veškerá logika serveru, tedy reaguje na parametry dotazu.

\bigbreak

Zpracování dotazu funguje pro všechny kombinace parametrů dotazu podobně. Vždy se data čtou z databáze a transformují se do formátu /gls{geojson}. Uvedˇme si na příkladu jak probíhá zpracování dotazu na zastávky daného spoje. Po té co obdržíme dotaz se z něj přečtou parametry. Pro dotaz na zastávky musí být parametr ve formátu \verb-get_stops.id_trip-. Parsování dotazu a volání příslušní interní funkce se provádí následovně.

\begin{code}[frame=none]
elif "stops" == request_body.split('.')[0]:
  response_body = json.dumps(self.get_stops(
    request_body[request_body.index('.')+1:]))
\end{code}

Funkce \verb-get_stops- přijímá identifikátor jízdy jako parametr a podle něj čte data z databáze pomocí \gls{sql} dotazu. Po přečtení dat vytváří mapu, která se snadno převede na řetězec ve formátu \gls{geojson}. Tělo funkce tedy vypadá takto:

\begin{code}[frame=none]
stops = self.database_connection.execute_fetchall("""
SELECT
  stops.lon,
  stops.lat,
  rides.departure_time,
  stops.stop_name
FROM rides
INNER JOIN stops ON rides.id_stop = stops.id_stop
WHERE rides.id_trip = %s
ORDER BY rides.shape_dist_traveled""",
(id_trip,)
)

stops_geojson = {}
stops_geojson["type"] = "FeatureCollection"
stops_geojson["features"] = []

for stop in stops:
stops_geojson["features"].append({
  "name": stop[3],
  "departure_time": stop[2].total_seconds(),
  "geometry": {
    "coordinates": [float(stop[0]), float(stop[1])]
  }
})

return stops_geojson
\end{code}



















%f


\chapter{Tesstování a evaluace}

V této kapitole je popsáno jak je celá aplikace otestována. Dále pak porovnání odhadů zpoždění se stávajícím řešení.

\section{Testování softwarového řešení}

Ko'd práce popsaný v kapitole \ref{chapter:implementace} je otestován unit testy. Propojení  tohoto softwaru s databází i zdrojem vstpuních dat je testováno integračními testy.

\subsection{Unit testy}

Unit testy testují spravnou funkčnost jednotlivých metod všech softwarových komponent této práce.

Pro ověření správné funkčnosti některých metod jsou vygenerována vstupní či výstupní data. To z důvodu, že tyto metody pracují z komplexní datovou strukturou nebo s velkým objemem dat, který není možno zadat jako vstupní přímo v ko'du testu, resp. je potřeba porovnat výstup tetované metody a ze stejných důvodů není možné uvádět výstupní hodnoty pro porovnání přímo v ko'du testu. Typickým příkladem takové vstupní struktury je model profilu jízdy, ptotože je potřeba otestovat funkce, které s takovým modelem pracují.

\subsection{Integrační testy}

TODO Popisovat co testují testy? Není v tom žádná složitá logika. Dále otázka k celému textu jak odkazovat na jednotlivé soubory s kodem?

\subsection{Výkonostní testy}

Všechny následující výkonostní testy jsou prováděny na osobním notebooku s technickými parametry uvedenými v tabulce \ref{table:hw}', kde všechny procesy aplikace včetně databáze běží paralelně.

\begin{center}
	\begin{table}[ht]
\centering
\begin{tabular}{|c|c|}
\hline
 Parametr & Hodnota \\ \hline \hline
 Procesor & 4x Intel(R) Core(TM) i7 CPU @ 2.70 GHz\\ \hline
 Pamětˇ & 16 GB DDR3 RAM  \\  \hline
 Rychlost zápisu na disk & 1000--3000 MB/Sec \\ \hline
 OS & macOS Big Sur\\ \hline
 MySQL & version 8.0.18\\ \hline
\end{tabular}
\label{table:hw}
\end{table}
\end{center}
\footnote{https://9to5mac.com/2016/11/01/the-late-2016-entry-level-13-macbook-pro-has-a-ridiculously-fast-ssd/}
\bigbreak

Pro potlačení zkreslení testů vlivem čekání na stažení dat z internetu jsou všechny data načítána z disku počítače.

\bigbreak

Testování stejně jako funkční a kvalitativní požadavky na práci vychází z analýzy vstupních dat uvedené v kapitole \ref{subsubsection:vstupni_soubory}.

\subsubsection{Zpracování dat}

Zpracování dat probíhá přečtením souboru s polohy vozidel a dále zpracovává každé vozidlo zvláštˇ. Přičemž pokud je vozdilo již nalezeno a jeho poloha se od poslední aktualizace změnila provedou se pouze 2 čtení z databáze, jeden záznam se aktualizuje a vloží se jeden nový záznam. Pokud je vozdilo již nalezeno a jeho poloha se od poslední aktualizace nezměnila provedese se pouze jedno čtení z databáze. Pokud ovšem je vozidlo obsluhující spoj nenalezeno musí se číst soubor s detailem daného spoje a všechna data se vkládají do databáze (jízdní řád včetně zastávek) navíc geografická lomená čára popisující jízdu se ukládá jako soubor.

\bigbreak

Jak je ale vidět na grafu \ref{fig:file_process_time} i pro nejvyšší množství vozidel (720) celé zpracování trvá nanejvýš 1.2 sekundy. Z toho plyne, že samotné zpracování dat není nijak časově náročné a vzhledem k 20sekundové periodě aktulizace dat máme velkou časovou rezervu. Rychlost zpracování jednoho vstupního souboru může více ovlivnit stahování dat z internetu, kde ale předpokládáme, že po většinu času nebude trvat stáhnout aktuální polohy vozidel déle než desítky milisekund.

\bigbreak

\begin{figure}
	\centering
  \includegraphics[width=0.7\linewidth]{../img/file_process_time}
  \caption{Průměrný čas zpracovávání daného počtu vozidel ze všech souborů se statickými daty. Světle modrá barva ohraničuje 95 \% interval spolehlivosti. Počty vozidel jsou vždy zaokrouhleny dolů na celé desítky.}
  \label{fig:file_process_time}
\end{figure}

\bigbreak

Jediné delší prodlení může nastat ve chvíli, kdy je potřeba stáhnout velké množství dodatečných informací o novém spoji. Na grafu \ref{fig:vehicle_pos_x_new_trips} je vidět, že až na jednotky vyjímek je počet nově nalezených spojů v jednom souboru nejvýše 20. Aplikace je ale naimplementována tak, aby se tyto informace stahovaly asynchroně a tedy čákání na stažení dat je co nejkratší.

\subsubsection{Konstrukce modelů}

Čtení dat potřebných pro trénování modelů z databáze, kde jsou data ze 4 dnů trvá přibližně 110 sekund. Čtení se totiž provádí komlikovaným \gls{sql} datazem uvedeným v kapitole \ref{subsection:cteni_dat}, ovšem na rychlost provedení tohoto dotazu i konstrukce modelů celkem neklademe žádné časové nároky, protože přepočítávání modelů je plánováno na čas nejmenšího zatížení systému, což bývá typicky v noci, kdy máme několik hodin pro výpočet.

\bigbreak

Dále ověřme, že výsledek dotazu nezahltí pamětˇ počítače, dotaz sice umožnˇuje čtení dat po stránkách, ale v implementaci se tato vlastnost nevyužívá. Určení velikosti objektu jazyka Python3 v paměti počítače není úplně trivální úloha, dobrý odhad nám, ale poskytne objektu na disk pomocí knihovny \verb-pickle-. Takto uložený objekt zabírá necelých 10 MB prostoru na disku.

\bigbreak



\section{Evaluace výsledků}

\subsection{Sestrojení modelů}

Po využití testovacích dat vzorků poloh vozidel zaznamenaných ve dnech 20.--24. února 2020 bylo podle krytérií, kterými jsou zejména vzdálenost zastávek a počet vzorků mezi nimi, sestrojeno celkem 1106 polynomiálních modelů. Z toho je 847 modelů pro pracovní dny, které jsou nejdůležitější. Přičemž celkový počet párů zastávek je 7230, ale zastávek ve vzdálenosti 1500 metrů\footnote{zvolená minimální vzdálenost mezi zastávkama, mezi kterýma má ještě smysl odhadovat zpoždění} je pouze 2142. Z toho vychází, že u 40 \% dvojic zastávek je dostatek dat, aby dával výpočet modelu smysl.

\bigbreak

U zbylých dvojic zastávek se využívá lineární model.

\subsection{Odhady zpoždění}

Z toho jak jsou definovány požadavky řešení v kapitole \ref{subsubsection:kvalitativni_pozadavky} pro změření kvality výsledků stačí porovnávat odhad zpoždění lineárního (původního) modelu a nového polynomiálního modelu. Přičemž odhad je lepší pokud má sekvence odhadů  z celé jízdy mezi dvojcí zastávek menší rozptyl.

\bigbreak

Podívejme se tedy na porovnání odhadů zpoždění novými modely profilů jízd se stávajícím řešení pracujícím s předpokladem, že vozidla jedou celou trasu mezi dvěma zastávkami konstantní rychlostí.

\bigbreak

Evaluaci výsledků budeme provádět s daty sesbíranými 20. 2. 2020, které použijeme jako trénovací data a s daty sesbíranými 21. 2. 2020, které použijeme jako testovací data. Toto je standardní postup pro hodnocení úspěšnosti predikcí modelů ve světě strojového učení. Modely nemohou být testovány na stejných datech jako na kterých byly trénovány, protože kdyby se trénovalo i testovalo na stejných datech, model by nemusel nic predikovat, ale stačilo by, aby si jen "zapamatoval" hodnotu z množiny trénovacích dat.

\bigbreak


\chapter*{Závěr}
\addcontentsline{toc}{chapter}{Závěr}

Nejprve jsme se seznámili s daty a po analýze dat se zkušeností a intuicí s vývojem dopravních situací jsme rozhodli, jestli je problém možné řešit a jaký dopad by mohl mít v reálném nasazení. Ačkoli požadavek na zlepšení odhadu nebo i předpovědi zpoždění vozidel \gls{vhd} se zdá naprosto přirozený, žádné z dosud existujících řešení zpracovávající real-time data se o nic takového nepokouší. O to víc je to překvapující s přihlédnutím k množství cestujících v Praze i jinde na světě.

\bigbreak

Dále jsem analyzovali jiné nástroje v České republice vizualizující polohy vozidel a definovali jsme si problém. Následně jsem si zadali samotné požadavky na dílo.

\bigbreak

Navrhli jsme procesy zpracování vstupních dat a jejich následné využití k pravděpodobnostnímu odhadu zpoždění. Také jsme navrhli algoritmus, který se používá v této práci pro odhat zpoždění a algoritmu který řeší komplikovanější situace, než s jakýma jsme se setkali v pražské dopravní síti. Ačkoli tento návrh alrogitmus neimplementujeme, může sloužit jako základní kámen pro další výzkum a obdobné aplikace. Součástí návrhu je design front-endu aplikace a databázové struktury.

\bigbreak

Zdrojový ko'd aplikace jsme zdokumentovali jako součást ko'du a logiku běhu softwaru jsme popsali v kapitole implementace. Rozděleně jsou popsány části zpracování dat, výpočet modelů sloužících k odhadu zpoždění a server-klient vizualizační aplikace. Pro serverou část jsme popsali jak se k ní připojit v případě vývoje jiné klienstké aplikace.

\bigbreak

Na závěr jsme se celou aplikaci otestovali a to od jednotlivých funkcích tříd až po aplikaci jako celek. Testování funkcí jsme provedli pomocí unit testů. Složitější celky včetně správné funkčnosti databáze jsme otestovali pomocí integračních testů. Server byl podroben zátěžovým testům, kdy jsme jej dotazovaly velkým množstvým paralelních dotazů. Rychlost zpracování dat byla měřena v bodě maximálního vytížení.

\bigbreak

Velkou část testování tvořilo ověření výsledků, tedy odhadů zpoždění. Z kterého vyplunulo, že v nezanedbatelném množství případů došlo opravdu k výraznému zlepšení. Zvolená metoda se však nedá uplatnit ve všech případech a to zejména z důvodu, že klasické řešení dosahuje dobrých výsledků a není tam tedy prostor ke zlepšení.

\section{Návrhy na zlepšení}

Tak jak je aplikace napsána je shopná samostatného běhu, avšak protože jsme nebyli schopní otestovat aplikaci v dlouhodobém nasazení je potřeba vyřešit několik problémů s tím souvisejících. První problém může být objemn dat, které je potřeba skladovat a s tím související spomalující se reakční doba databáze. V takovém případě navrhujeme historická data skladovat na odděleném místě od databáze, kam ukládáme aktuální data a do které se dotazuje server. Dále je potřeba také určit správnou periodu přepočítání modelů a jaké data a jak stará pro ně použít.

\bigbreak

Protože jsme se v návrhu rozhodli rozdělit vstupní data pro výpočet modelů na zaznamená v pracovní den a víkendový den nabízí se zlepšení do dní pracovního volna započíst i státní svátky nebo ještě lépe použít data pro výpočet modelu jen z jednoho dne v týdnu a vázat tak model na den v týdnu.

\bigbreak

Vzhledem k tomu, že v této práci pracujeme s geografickými daty, bylo by vhodnější použít jinou \gls{sql} databázi, protože MySQL databáze není příliš uzpůsobená pro ukládání tohoto typu dat. Vhodnější implementace databáze se nabízí PostgreSQL\footnote{https://www.postgresql.org} s rozšířením PostGIS\footnote{https://postgis.net}.

\bigbreak

Jako poslední návrh na zlepšení je samotné vylepšení modelů popisujících profily jízd. Atˇ už zvolením mnodelu jiného typu nebo zcela odlišného přístupu k problému. Aplikace je nastavena tak, aby se při vylepšování modelů nemusely měnit jiné části aplikace. 


%%% Seznam použité literatury
\include{literatura}

%%% Obrázky v bakalářské práci
%%% (pokud jich je malé množství, obvykle není třeba seznam uvádět)
\listoffigures

%%% Tabulky v bakalářské práci (opět nemusí být nutné uvádět)
%%% U matematických prací může být lepší přemístit seznam tabulek na začátek práce.
\listoftables

%%% Použité zkratky v bakalářské práci (opět nemusí být nutné uvádět)
%%% U matematických prací může být lepší přemístit seznam zkratek na začátek práce.
\chapwithtoc{Seznam použitých zkratek}
\printnoidxglossaries

%%% Přílohy k bakalářské práci, existují-li. Každá příloha musí být alespoň jednou
%%% odkazována z vlastního textu práce. Přílohy se číslují.
%%%
%%% Do tištěné verze se spíše hodí přílohy, které lze číst a prohlížet (dodatečné
%%% tabulky a grafy, různé textové doplňky, ukázky výstupů z počítačových programů,
%%% apod.). Do elektronické verze se hodí přílohy, které budou spíše používány
%%% v elektronické podobě než čteny (zdrojové kódy programů, datové soubory,
%%% interaktivní grafy apod.). Elektronické přílohy se nahrávají do SISu a lze
%%% je také do práce vložit na CD/DVD. Povolené formáty souborů specifikuje
%%% opatření rektora č. 72/2017.
\appendix
\chapter{Přílohy}

\section{První příloha}

\openright
\end{document}
