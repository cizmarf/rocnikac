
\chapter{Tesstování a evaluace}

V této kapitole je popsáno jak je celá aplikace otestována. Dále pak porovnání odhadů zpoždění se stávajícím řešení.

\section{Testování softwarového řešení}

Ko'd práce popsaný v kapitole \ref{chapter:implementace} je otestován unit testy. Propojení  tohoto softwaru s databází i zdrojem vstpuních dat je testováno integračními testy.

\subsection{Unit testy}

Unit testy testují spravnou funkčnost jednotlivých metod všech softwarových komponent této práce.

Pro ověření správné funkčnosti některých metod jsou vygenerována vstupní či výstupní data. To z důvodu, že tyto metody pracují z komplexní datovou strukturou nebo s velkým objemem dat, který není možno zadat jako vstupní přímo v ko'du testu, resp. je potřeba porovnat výstup tetované metody a ze stejných důvodů není možné uvádět výstupní hodnoty pro porovnání přímo v ko'du testu. Typickým příkladem takové vstupní struktury je model profilu jízdy, ptotože je potřeba otestovat funkce, které s takovým modelem pracují.

\subsection{Integrační testy}

TODO Popisovat co testují testy? Není v tom žádná složitá logika. Dále otázka k celému textu jak odkazovat na jednotlivé soubory s kodem?

\section{Evaluace výsledků}

\subsection{Sestrojení modelů}

Po využití testovacích dat vzorků poloh vozidel zaznamenaných ve dnech 20.--24. února 2020 bylo podle krytérií, kterými jsou zejména vzdálenost zastávek a počet vzorků mezi nimi, sestrojeno celkem 1106 polynomiálních modelů. Z toho je 847 modelů pro pracovní dny, které jsou nejdůležitější. Přičemž celkový počet párů zastávek je 7230, ale zastávek ve vzdálenosti 1500 metrů\footnote{zvolená minimální vzdálenost mezi zastávkama, mezi kterýma má ještě smysl odhadovat zpoždění} je pouze 2142. Z toho vychází, že u 40 \% dvojic zastávek je dostatek dat, aby dával výpočet modelu smysl.

\bigbreak

U zbylých dvojic zastávek se využívá lineární model.

\subsection{Odhady zpoždění}

Z toho jak jsou definovány požadavky řešení v kapitole \ref{subsubsection:kvalitativni_pozadavky} pro změření kvality výsledků stačí porovnávat odhad zpoždění lineárního (původního) modelu a nového polynomiálního modelu. Přičemž odhad je lepší pokud má sekvence odhadů  z celé jízdy mezi dvojcí zastávek menší rozptyl.

\bigbreak

Podívejme se tedy na porovnání odhadů zpoždění novými modely profilů jízd se stávajícím řešení pracujícím s předpokladem, že vozidla jedou celou trasu mezi dvěma zastávkami konstantní rychlostí.

\bigbreak

Evaluaci výsledků budeme provádět s daty sesbíranými 20. 2. 2020, které použijeme jako trénovací data a s daty sesbíranými 21. 2. 2020, které použijeme jako testovací data. Toto je standardní postup pro hodnocení úspěšnosti predikcí modelů ve světě strojového učení. Modely nemohou být testovány na stejných datech jako na kterých byly trénovány, protože kdyby se trénovalo i testovalo na stejných datech, model by nemusel nic predikovat, ale stačilo by, aby si jen "zapamatoval" hodnotu z množiny trénovacích dat.

\bigbreak
