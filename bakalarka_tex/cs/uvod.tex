\chapter*{Úvod}
\addcontentsline{toc}{chapter}{Úvod}

Městská hromadná doprava v Praze a Středočeském kraji je jeden z hlavním pilířů přepravy osob na tomto území. Jejím rozsahem a důležitostí se přímo dotýká každého z nás a její fungování do značné míry ovlivňuje naše konání v krátkém i dlouhém časovém horizontu.

\bigbreak

Každého cestujícího v přepravě jistě někdy trápilo zpoždění svého spoje. To člověka přivádí k myšlenkám jestli by nebylo možné určit s jakou pravidelností, pokud s nějakou, takové zpoždění vznikají. A jestli by nemohl být informován za včasu o vzniklé anomálii.

\bigbreak

Ve vymezené oblasti operuje spousta soukromých i městských dopravců. Ti kteří spadají do naší zájmové oblasti zastřešuje organizace \gls{ropid}, která objednává jednotlivé spoje. Pro tuto práci je však důležité, že tato organizace zadala jednotlivým dopravcům vysílat aktuálním polohy jejich vozů. Tato data jsou přes zprostředkovatele zveřejňována na pražské datové platformě zvané Golemio, jež je ve správě společností Operátor ICT, a. s., která je vlastněná hlavním městem Praha.

\bigbreak

Nicméně v době návrhu práce, kvůli právním komplikacím, nebyly k dispozici real-time data z majoritního dopravce na území Prahy Dopravní podnik hl. m. Prahy. Vzhledem k povaze této práce avšak tyto data nenabývají takové důležitosti jako data od dopravců operujících mimo Prahu. Vzhledem k tomu, že zbylí dopravci využívají převážně autobusy k přepravě cestujících, bude práce vypracována pouze s ohledem na autobusovou dopravu.

\bigbreak

Práce se tedy pokusí využít dostupná otevřená data k získání infomarcí o zpoždění spojů na trase. Řešení ovšem není pouze založeno na real-tim datach, ale využívá také statická data o jízdních řádech nebo zastávkách hromadné dopravy a také mapové podklady.

\bigbreak

Vzhledem k disponování daty o aktuálních polohách vozidel \gls{mhd} se nabízí jejich využití tak, že budou vynesena do mapy a tím vznikne vizuálně přívětivé uživatelské prostředí pro prohlížení aktuálního stavu sítě vozidel. Proto práce navrhuje a implementuje uživatelskou aplikaci, která tyto vozidla zobrazí a bude itergagovat s uživatelem tak, že po na uživatelskou žádost zobrazí additivní infomace o daném spoji nebo vybvrané zastávce.


TODO zapracovat odstavec: podle získaných dat se v pracovní den vypravý přibližně (TODO spocitat pocet autobusu denne) autobusových spojů.
